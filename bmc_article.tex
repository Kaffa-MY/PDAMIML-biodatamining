%% BioMed_Central_Tex_Template_v1.06
%%                                      %
%  bmc_article.tex            ver: 1.06 %
%                                       %

%%IMPORTANT: do not delete the first line of this template
%%It must be present to enable the BMC Submission system to
%%recognise this template!!

%%%%%%%%%%%%%%%%%%%%%%%%%%%%%%%%%%%%%%%%%
%%                                     %%
%%  LaTeX template for BioMed Central  %%
%%     journal article submissions     %%
%%                                     %%
%%          <8 June 2012>              %%
%%                                     %%
%%                                     %%
%%%%%%%%%%%%%%%%%%%%%%%%%%%%%%%%%%%%%%%%%


%%%%%%%%%%%%%%%%%%%%%%%%%%%%%%%%%%%%%%%%%%%%%%%%%%%%%%%%%%%%%%%%%%%%%
%%                                                                 %%
%% For instructions on how to fill out this Tex template           %%
%% document please refer to Readme.html and the instructions for   %%
%% authors page on the biomed central website                      %%
%% http://www.biomedcentral.com/info/authors/                      %%
%%                                                                 %%
%% Please do not use \input{...} to include other tex files.       %%
%% Submit your LaTeX manuscript as one .tex document.              %%
%%                                                                 %%
%% All additional figures and files should be attached             %%
%% separately and not embedded in the \TeX\ document itself.       %%
%%                                                                 %%
%% BioMed Central currently use the MikTex distribution of         %%
%% TeX for Windows) of TeX and LaTeX.  This is available from      %%
%% http://www.miktex.org                                           %%
%%                                                                 %%
%%%%%%%%%%%%%%%%%%%%%%%%%%%%%%%%%%%%%%%%%%%%%%%%%%%%%%%%%%%%%%%%%%%%%

%%% additional documentclass options:
%  [doublespacing]
%  [linenumbers]   - put the line numbers on margins

%%% loading packages, author definitions

%\documentclass[twocolumn]{bmcart}% uncomment this for twocolumn layout and comment line below
\documentclass{bmcart}

%%% Load packages
%\usepackage{amsthm,amsmath}
%\RequirePackage{natbib}
%\RequirePackage{hyperref}
\usepackage[utf8]{inputenc} %unicode support
%\usepackage[applemac]{inputenc} %applemac support if unicode package fails
%\usepackage[latin1]{inputenc} %UNIX support if unicode package fails


%%%%%%%%%%%%%%%%%%%%%%%%%%%%%%%%%%%%%%%%%%%%%%%%%
%%                                             %%
%%  If you wish to display your graphics for   %%
%%  your own use using includegraphic or       %%
%%  includegraphics, then comment out the      %%
%%  following two lines of code.               %%
%%  NB: These line *must* be included when     %%
%%  submitting to BMC.                         %%
%%  All figure files must be submitted as      %%
%%  separate graphics through the BMC          %%
%%  submission process, not included in the    %%
%%  submitted article.                         %%
%%                                             %%
%%%%%%%%%%%%%%%%%%%%%%%%%%%%%%%%%%%%%%%%%%%%%%%%%


\def\includegraphic{}
\def\includegraphics{}



%%% Put your definitions there:
\startlocaldefs
\endlocaldefs


%%% Begin ...
\begin{document}

%%% Start of article front matter
\begin{frontmatter}

\begin{fmbox}
\dochead{Research}

%%%%%%%%%%%%%%%%%%%%%%%%%%%%%%%%%%%%%%%%%%%%%%
%%                                          %%
%% Enter the title of your article here     %%
%%                                          %%
%%%%%%%%%%%%%%%%%%%%%%%%%%%%%%%%%%%%%%%%%%%%%%

\title{PDAMIML:A Multi-Instance Multi-Label Learning Framework for Protein Domain Annotation}

%%%%%%%%%%%%%%%%%%%%%%%%%%%%%%%%%%%%%%%%%%%%%%
%%                                          %%
%% Enter the authors here                   %%
%%                                          %%
%% Specify information, if available,       %%
%% in the form:                             %%
%%   <key>={<id1>,<id2>}                    %%
%%   <key>=                                 %%
%% Comment or delete the keys which are     %%
%% not used. Repeat \author command as much %%
%% as required.                             %%
%%                                          %%
%%%%%%%%%%%%%%%%%%%%%%%%%%%%%%%%%%%%%%%%%%%%%%

\author[
   addressref={aff1},                   % id's of addresses, e.g. {aff1,aff2}
   corref={aff1},                       % id of corresponding address, if any
   noteref={n1},                        % id's of article notes, if any
   email={jane.e.doe@cambridge.co.uk}   % email address
]{\inits{JE}\fnm{Jane E} \snm{Doe}}
\author[
   addressref={aff1,aff2},
   email={john.RS.Smith@cambridge.co.uk}
]{\inits{JRS}\fnm{John RS} \snm{Smith}}

%%%%%%%%%%%%%%%%%%%%%%%%%%%%%%%%%%%%%%%%%%%%%%
%%                                          %%
%% Enter the authors' addresses here        %%
%%                                          %%
%% Repeat \address commands as much as      %%
%% required.                                %%
%%                                          %%
%%%%%%%%%%%%%%%%%%%%%%%%%%%%%%%%%%%%%%%%%%%%%%

\address[id=aff1]{%                           % unique id
  \orgname{Department of Zoology, Cambridge}, % university, etc
  \street{Waterloo Road},                     %
  %\postcode{}                                % post or zip code
  \city{London},                              % city
  \cny{UK}                                    % country
}
\address[id=aff2]{%
  \orgname{Marine Ecology Department, Institute of Marine Sciences Kiel},
  \street{D\"{u}sternbrooker Weg 20},
  \postcode{24105}
  \city{Kiel},
  \cny{Germany}
}

%%%%%%%%%%%%%%%%%%%%%%%%%%%%%%%%%%%%%%%%%%%%%%
%%                                          %%
%% Enter short notes here                   %%
%%                                          %%
%% Short notes will be after addresses      %%
%% on first page.                           %%
%%                                          %%
%%%%%%%%%%%%%%%%%%%%%%%%%%%%%%%%%%%%%%%%%%%%%%

\begin{artnotes}
%\note{Sample of title note}     % note to the article
\note[id=n1]{Equal contributor} % note, connected to author
\end{artnotes}

\end{fmbox}% comment this for two column layout

%%%%%%%%%%%%%%%%%%%%%%%%%%%%%%%%%%%%%%%%%%%%%%
%%                                          %%
%% The Abstract begins here                 %%
%%                                          %%
%% Please refer to the Instructions for     %%
%% authors on http://www.biomedcentral.com  %%
%% and include the section headings         %%
%% accordingly for your article type.       %%
%%                                          %%
%%%%%%%%%%%%%%%%%%%%%%%%%%%%%%%%%%%%%%%%%%%%%%

\begin{abstractbox}

\begin{abstract} % abstract
\parttitle{First part title} %if any
Text for this section.

\parttitle{Second part title} %if any
Text for this section.
\end{abstract}

%%%%%%%%%%%%%%%%%%%%%%%%%%%%%%%%%%%%%%%%%%%%%%
%%                                          %%
%% The keywords begin here                  %%
%%                                          %%
%% Put each keyword in separate \kwd{}.     %%
%%                                          %%
%%%%%%%%%%%%%%%%%%%%%%%%%%%%%%%%%%%%%%%%%%%%%%

\begin{keyword}
\kwd{domain annotation}
\kwd{multi-instance multi-label}
\kwd{SVM}
\kwd{auto covariance transformation}
\end{keyword}

% MSC classifications codes, if any
%\begin{keyword}[class=AMS]
%\kwd[Primary ]{}
%\kwd{}
%\kwd[; secondary ]{}
%\end{keyword}

\end{abstractbox}
%
%\end{fmbox}% uncomment this for twcolumn layout

\end{frontmatter}

%%%%%%%%%%%%%%%%%%%%%%%%%%%%%%%%%%%%%%%%%%%%%%
%%                                          %%
%% The Main Body begins here                %%
%%                                          %%
%% Please refer to the instructions for     %%
%% authors on:                              %%
%% http://www.biomedcentral.com/info/authors%%
%% and include the section headings         %%
%% accordingly for your article type.       %%
%%                                          %%
%% See the Results and Discussion section   %%
%% for details on how to create sub-sections%%
%%                                          %%
%% use \cite{...} to cite references        %%
%%  \cite{koon} and                         %%
%%  \cite{oreg,khar,zvai,xjon,schn,pond}    %%
%%  \nocite{smith,marg,hunn,advi,koha,mouse}%%
%%                                          %%
%%%%%%%%%%%%%%%%%%%%%%%%%%%%%%%%%%%%%%%%%%%%%%

%%%%%%%%%%%%%%%%%%%%%%%%% start of article main body
% <put your article body there>

%%%%%%%%%%%%%%%%
%% Background %%
%%
\section*{Introduction}
One of the most challenging and intriguing problems in the postgenomic era is the characterization
of the biochemical functions of proteins. Accurate computational assignment of protein function is becoming a useful resource for
both the community at large and the curators that eventually assign function to proteins. It is known that domains appear either singly or in combination with other domains as building blocks in a protein\\cite{Apic2001}\\cite{Wang2006}. Domains play important roles in the process of protein-protein interactions which determine the function of the protein\\cite{Bork1999}. Identifying functions of domains provides key clues in the annotation of protein functions. Since the number of domains is relatively finite while the number of unannotated proteins is considerably large, it will be much easier to infer the function of proteins if the functions of their component domains are determined. Therefore, there is a great need to develop accurate computational methods for domain function annotation.

Traditional work of annotation of functions is based on the physicochemical properties of special structure and primarily conducted manually, which is a time-consuming, low-efficient and experience-dependent process. Moreover, with the rapid growth of proteins and domains, manual annotation becomes increasing infeasible. To deal with this problem, researchers introduced computational methods into the field of function annotation, in order to simplify and improve the work.

Despite a variety of the strategies and methods have used in identifying protein functions, only a few researchers have paid attentions to the field of domain annotation. For which only a relatively small number of domains have been annotated until now, several innovative methods have been proposed to predict domain function. Schug \\textit{et al.}\\cite{Schug2002} described a heuristic algorithm for associating Gene Ontology (GO)\\cite{Ashburner2000} defined molecular functions to proteins as listed in the ProDom\\cite{Corpet2000} and CDD\\cite{Marchler2003} databases. The algorithm generates rules for function-domain associations based on the intersection of functions assigned to gene products by the GO consortium that contain ProDom and/or CDD domains at varying levels of sequence similarity. Lu \\textit{et al.}\\cite{Lu2004} utilized protein-domain mapping (P2D) features to construct a logistic regression model to investigate the association rules between target domains and GO terms.

Recently, domain-domain interaction information has provided a new way to predict domain function, since it's rational to assume that interacting domains have a high probability to share similar functions. Deng \\textit{et al.}\\cite{Deng2002} applied a Maximum Likelihood Estimation method to infer interaction domains that are consistent with the observed protein-protein interaction. Their method showed robustness in analyzing incomplete data sets and dealing with various experimental errors. Moreover, some researchers utilize domain coexisting features to predict domain function. For example, Wang \\textit{et al.}\\cite{Wang2007} gave a general framework to predict protein interactions by considering the information of both multi-domains and multi organisms, which can also be applied to identify cooperative domains, further to annotate functions of domains.

Based on the previous works, Zhao and Wang\\cite{Zhao2008} designed two methods, the threshold-based classification method and the support vector machine method, for protein domain function prediction by integrating heterogeneous information sources, including protein-domain mapping features(P2D)\\cite{Lu2004}, domain-domain interactions(DDI)\\cite{Deng2002}, and domain coexisting features(CDD)\\cite{Wang2007}, improving not only prediction accuracy but also annotation reliability. During these two methods, the threshold-based classification method outperforms the support vector machine method according to Zhao's\\cite{Zhao2008} experiments.

In those traditional learning formalizations, each domain is represented by an instance (or feature vector) and associated with single GO term. Although the above single instance and single label formalization is prevailing and successful, it is not an appropriate model for the actual situation of domain function annotation. In fact, each domain usually exists in multiple proteins, which can be described by a feature vector, and the domain can belong to multiple categories since it is associated with some different functions.

In this paper, we propose a novel Multi-Instance Multi-Label Learning (MIML)\\cite{Zhou2012} based framework, PDAMIML, to predict functions of protein domains accurately. PDAMIML combines MIML model, support vector machine\\cite{Vapnik1998} and auto-cross covariance\\cite{Wold1993} to overcome the multi-label classification problem and effectively utilize the features of Position-Specific Scoring Matrix (PSSM)\\cite{Stephen1997}. Furthermore, we design an ensemble method, PDAMIML-ensemble, that integrates PDAMIML and other two eminent threshold-based approaches (CDD and P2D\\cite{Zhao2008}) with majority voting strategy. Our experimental results show that PDAMIML-ensemble significantly outperforms the state-of-the-art domain annotation approaches.

\section*{Materials and Methods}
\subsection*{Datasets}
Relationships between proteins and domains are obtained from InterPro Database\\cite{Sarah2011}. The function annotations of domains are generated from GOA Database\\cite{Evelyn2003}. InterPro contains three main entities: proteins, signatures and entries. The signatures from InterPro comes from 11 member databases, including Pfam\\cite{Bateman2002}, ProDom\\cite{Corpet1999}, etc. The protein-domain-function dataset contains 13137 proteins, 5748 domains, and 2535 GO terms. Around 76\% domains are annotated with more than one GO term, and every domain has 2.5 annotations on average. We choose the top 100 domain with the most GO terms as the data set, and then filter sparse labels which are inconvenient for study, select the most frequent 10 GO terms as target labels out of the total 188 labels attached to the top 100 domains according to their frequency. The diagram of frequency of GO terms is showed in Fig.\\ref{goterms} and the definition of the 10 GO terms we select is demonstrated in TABLE \\ref{definition}. Each protein can be viewed as an instance and each domain is represented as a bag of instances.
\subsubsection*{Sub-sub heading for section}
Text for this sub-sub-heading \ldots
\paragraph*{Sub-sub-sub heading for section}
Text for this sub-sub-sub-heading \ldots
In this section we examine the growth rate of the mean of $Z_0$, $Z_1$ and $Z_2$. In
addition, we examine a common modeling assumption and note the
importance of considering the tails of the extinction time $T_x$ in
studies of escape dynamics.
We will first consider the expected resistant population at $vT_x$ for
some $v>0$, (and temporarily assume $\alpha=0$)
%
\[
 E \bigl[Z_1(vT_x) \bigr]= E
\biggl[\mu T_x\int_0^{v\wedge
1}Z_0(uT_x)
\exp \bigl(\lambda_1T_x(v-u) \bigr)\,du \biggr].
\]
%
If we assume that sensitive cells follow a deterministic decay
$Z_0(t)=xe^{\lambda_0 t}$ and approximate their extinction time as
$T_x\approx-\frac{1}{\lambda_0}\log x$, then we can heuristically
estimate the expected value as
%
\begin{eqnarray}\label{eqexpmuts}
E\bigl[Z_1(vT_x)\bigr] &=& \frac{\mu}{r}\log x
\int_0^{v\wedge1}x^{1-u}x^{({\lambda_1}/{r})(v-u)}\,du
\nonumber\\
&=& \frac{\mu}{r}x^{1-{\lambda_1}/{\lambda_0}v}\log x\int_0^{v\wedge
1}x^{-u(1+{\lambda_1}/{r})}\,du
\nonumber\\
&=& \frac{\mu}{\lambda_1-\lambda_0}x^{1+{\lambda_1}/{r}v} \biggl(1-\exp \biggl[-(v\wedge1) \biggl(1+
\frac{\lambda_1}{r}\biggr)\log x \biggr] \biggr).
\end{eqnarray}
%
Thus we observe that this expected value is finite for all $v>0$ (also see \cite{koon,khar,zvai,xjon,marg}).



%%%%%%%%%%%%%%%%%%%%%%%%%%%%%%%%%%%%%%%%%%%%%%
%%                                          %%
%% Backmatter begins here                   %%
%%                                          %%
%%%%%%%%%%%%%%%%%%%%%%%%%%%%%%%%%%%%%%%%%%%%%%

\begin{backmatter}

\section*{Competing interests}
  The authors declare that they have no competing interests.

\section*{Author's contributions}
    Text for this section \ldots

\section*{Acknowledgements}
  Text for this section \ldots
%%%%%%%%%%%%%%%%%%%%%%%%%%%%%%%%%%%%%%%%%%%%%%%%%%%%%%%%%%%%%
%%                  The Bibliography                       %%
%%                                                         %%
%%  Bmc_mathpys.bst  will be used to                       %%
%%  create a .BBL file for submission.                     %%
%%  After submission of the .TEX file,                     %%
%%  you will be prompted to submit your .BBL file.         %%
%%                                                         %%
%%                                                         %%
%%  Note that the displayed Bibliography will not          %%
%%  necessarily be rendered by Latex exactly as specified  %%
%%  in the online Instructions for Authors.                %%
%%                                                         %%
%%%%%%%%%%%%%%%%%%%%%%%%%%%%%%%%%%%%%%%%%%%%%%%%%%%%%%%%%%%%%

% if your bibliography is in bibtex format, use those commands:
\bibliographystyle{bmc-mathphys} % Style BST file
\bibliography{bmc_article}      % Bibliography file (usually '*.bib' )

% or include bibliography directly:
% \begin{thebibliography}
% \bibitem{b1}
% \end{thebibliography}

%%%%%%%%%%%%%%%%%%%%%%%%%%%%%%%%%%%
%%                               %%
%% Figures                       %%
%%                               %%
%% NB: this is for captions and  %%
%% Titles. All graphics must be  %%
%% submitted separately and NOT  %%
%% included in the Tex document  %%
%%                               %%
%%%%%%%%%%%%%%%%%%%%%%%%%%%%%%%%%%%

%%
%% Do not use \listoffigures as most will included as separate files

\section*{Figures}
  \begin{figure}[h!]
  \caption{\csentence{Sample figure title.}
      A short description of the figure content
      should go here.}
      \end{figure}

\begin{figure}[h!]
  \caption{\csentence{Sample figure title.}
      Figure legend text.}
      \end{figure}

%%%%%%%%%%%%%%%%%%%%%%%%%%%%%%%%%%%
%%                               %%
%% Tables                        %%
%%                               %%
%%%%%%%%%%%%%%%%%%%%%%%%%%%%%%%%%%%

%% Use of \listoftables is discouraged.
%%
\section*{Tables}
\begin{table}[h!]
\caption{Sample table title. This is where the description of the table should go.}
      \begin{tabular}{cccc}
        \hline
           & B1  &B2   & B3\\ \hline
        A1 & 0.1 & 0.2 & 0.3\\
        A2 & ... & ..  & .\\
        A3 & ..  & .   & .\\ \hline
      \end{tabular}
\end{table}

%%%%%%%%%%%%%%%%%%%%%%%%%%%%%%%%%%%
%%                               %%
%% Additional Files              %%
%%                               %%
%%%%%%%%%%%%%%%%%%%%%%%%%%%%%%%%%%%

\section*{Additional Files}
  \subsection*{Additional file 1 --- Sample additional file title}
    Additional file descriptions text (including details of how to
    view the file, if it is in a non-standard format or the file extension).  This might
    refer to a multi-page table or a figure.

  \subsection*{Additional file 2 --- Sample additional file title}
    Additional file descriptions text.


\end{backmatter}
\end{document}
